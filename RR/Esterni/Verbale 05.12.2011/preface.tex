%%!!ATTENZIONE!!
%QUESTO FILE E' UN ESEMPIO. COPIATE ALL'INTERNO DELLA CARTELLA DEL DOCUMENTO SU CUI STATE LAVORANDO E MODIFICATE
%% NON MODIFICARE QUESTE IMPOSTAZIONI
\date{}
\usepackage[latin9]{inputenc}
\usepackage{fancyhdr}
\usepackage{ucs}
\usepackage{lastpage}
\usepackage{longtable}
\usepackage{hyperref}
\pagestyle{fancy}
\fancyhead{}
\fancyfoot{}
\newcommand{\HRule}{\rule{\linewidth}{0.5mm}}
\fancyhead[RE, RO]{\doctitle \versiondoc - \TeamCommitted}
\fancyfoot[CE, CO]{\thepage\ di \pageref{LastPage}}

%%Comandi particolari per i nomi delle figure. Chiamare come una funzione LaTeX
\newcommand{\TeamCommitted}{\emph{Team Committed }}
\newcommand{\respProg}{\emph{Responsabile di Progetto }}
\newcommand{\ammProg}{\emph{Amministratore di Progetto }}
\newcommand{\analProg}{\emph{Analista }}
\newcommand{\verifProg}{\emph{Verificatore }}
\newcommand{\programProg}{\emph{Programmatore }}
\newcommand{\progetProg}{\emph{Progettista }}
\newcommand{\proponProg}{\emph{Proponente }}
\newcommand{\commitProg}{\emph{Committente }}
\newcommand{\propProg}{\emph{Dr. Amir Baldissera }}
\newcommand{\commProg}{\emph{Prof. Tullio Vardanega }}
\newcommand{\nameproject}{\emph{Progetto Gamification}}

%%DA QUI IN POI POTETE MODIFICARE

%% INSERIRE QUI IL NOME DEL DOCUMENTO (INSERITE SEMPRE UNO SPAZIO ALLA FINE DEL NOME)
\newcommand{\doctitle}{Verbale }

%% INSERIRE QUI LA VERSIONE ATTUALE DEL DOCUMENTO (INSERITE SEMPRE UNO SPAZIO ALLA FINE DELLA VERSIONE)
\newcommand{\versiondoc}{05/12/2011 }

%%INSERITE QUI LA DATA DI COMPILAZIONE FINALE DEL DOCUMENTO
\newcommand{\datared}{06/12/2011}

%%INSERIRE QUI IL/I REDATTORI
\newcommand{\redattore}{Giorgio Maggiolo}

%%INSERIRE IL/I NOME DEI VERIFICATORI CHE HANNO VERIFICATO IL DOCUMENTO
\newcommand{\verificatori}{Giorgio Maggiolo}

%%INSERIRE IL NOME DI CHI HA APPROVATO IL DOCUMENTO
\newcommand{\approvazione}{Marco Begolo}

%%INSERIRE LA TIPOLOGIA DI USO DEL DOCUMENTO [Interno/Esterno]
\newcommand{\usodoc}{Esterno}

%%INSERIRE LA LISTA DI DISTRIBUZIONE DEL DOCUMENTO
\newcommand{\listadistr}{\begin{itemize}
\item \TeamCommitted
\item \propProg
\end{itemize}}

%%INSERIRE IL SOMMARIO DEL DOCUMENTO
\newcommand{\testosommario}{Verbale della riunione tenutasi il 05/12/2011 fra il \TeamCommitted e il \proponProg \propProg}

